\documentclass[conference]{IEEEtran}
\usepackage{times}

% numbers option provides compact numerical references in the text. 
\usepackage[numbers]{natbib}
\usepackage{multicol}
\usepackage[bookmarks=true]{hyperref}
\usepackage{mathtools}
\usepackage{amsmath}
\usepackage{amsfonts}
\usepackage{amssymb}
\usepackage{siunitx}
\usepackage{optidef}
\usepackage{algorithmicx}
\usepackage{xcolor}
\usepackage{algorithm,algpseudocode}
\usepackage{physics}  % for norm command
% \usepackage{gensymb}  % for degree symbol
\usepackage{bm}  % for bold symbols 
\usepackage{booktabs}
\usepackage{pifont}  % for x mark
\usepackage{pgfplots}
\pgfplotsset{compat=1.15,
	legend style={font=\footnotesize},
}
\usepackage{tikzscale}

\newcommand{\half}{\frac{1}{2}}
\newcommand{\R}{\mathbb{R}}
\newcommand{\Q}{\mathbb{S}^3}
\newcommand{\skewmat}[1]{[#1]^\times}

\newcommand{\rmap}{\varphi}
\newcommand{\invrmap}{\varphi^{-1}}

\newcommand{\dR}{\delta \mathcal{R}}
\newcommand{\rot}{ \mathcal{R} }
\newcommand{\dq}{\delta q}
\newcommand{\q}{\textbf{q}}
\newcommand{\eq}{_\text{eq}}
\newcommand{\traj}[2][N]{#2_{0:{#1}}}
\newcommand{\pass}{{\color{green} \checkmark}}
\newcommand{\fail}{{\color{red} \ding{55}}}

\pdfinfo{
   /Author (Homer Simpson)
   /Title  (Robots: Our new overlords)
   /CreationDate (D:20101201120000)
   /Subject (Robots)
   /Keywords (Robots;Overlords)
}

\begin{document}

% paper title
\title{Planning and Control with Attitude}

\author{\authorblockN{Michael Shell}
\authorblockA{School of Electrical and\\Computer Engineering\\
Georgia Institute of Technology\\
Atlanta, Georgia 30332--0250\\
Email: mshell@ece.gatech.edu}
\and
\authorblockN{Homer Simpson}
\authorblockA{Twentieth Century Fox\\
Springfield, USA\\
Email: homer@thesimpsons.com}
\and
\authorblockN{James Kirk\\ and Montgomery Scott}
\authorblockA{Starfleet Academy\\
San Francisco, California 96678-2391\\
Telephone: (800) 555--1212\\
Fax: (888) 555--1212}}


% avoiding spaces at the end of the author lines is not a problem with
% conference papers because we don't use \thanks or \IEEEmembership


% for over three affiliations, or if they all won't fit within the width
% of the page, use this alternative format:
% 
%\author{\authorblockN{Michael Shell\authorrefmark{1},
%Homer Simpson\authorrefmark{2},
%James Kirk\authorrefmark{3}, 
%Montgomery Scott\authorrefmark{3} and
%Eldon Tyrell\authorrefmark{4}}
%\authorblockA{\authorrefmark{1}School of Electrical and Computer Engineering\\
%Georgia Institute of Technology,
%Atlanta, Georgia 30332--0250\\ Email: mshell@ece.gatech.edu}
%\authorblockA{\authorrefmark{2}Twentieth Century Fox, Springfield, USA\\
%Email: homer@thesimpsons.com}
%\authorblockA{\authorrefmark{3}Starfleet Academy, San Francisco, California 96678-2391\\
%Telephone: (800) 555--1212, Fax: (888) 555--1212}
%\authorblockA{\authorrefmark{4}Tyrell Inc., 123 Replicant Street, Los Angeles, California 90210--4321}}


\maketitle

\begin{abstract}
Planning and controlling trajectories for floating-base robotic systems that experience large attitude changes is challenging due to the nontrivial group structure of 3D rotations. This paper introduces an accessible and unified approach for tracking control and optimization-based planning on the space of rotations. The methodology is used to derive an extension of the Linear-Quadratic Regulator (LQR) to systems with arbitrary attitudes, which we call Multiplicative LQR (MLQR). We compare MLQR to a specialized tracking controller designed for the SE(3) group, and derive an iterative variant of MLQR to optimize trajectories for a variety of robotic systems. We provide benchmark comparisons between several of the most common attitude representations used for motion planning and control and find that the combination of unit quaternion state variables with Rodrigues parameters to represent attitude errors provides an excellent combination of performance and algorithmic simplicity. 
\end{abstract}

\IEEEpeerreviewmaketitle

\end{document}


